\documentclass{article}
\usepackage[utf8]{inputenc}
\usepackage[spanish]{babel}
\usepackage{listings}
\usepackage{graphicx}
\graphicspath{ {images/} }
\usepackage{cite}

\begin{document}

\begin{titlepage}
    \begin{center}
        \vspace*{1cm}
            
        \Huge
        \textbf{Nociones de la memoria del computador}
            
        \vspace{0.5cm}
        \LARGE
        
            
        \vspace{5.5cm}
            
        \textbf{Luis Miguel Martinez Ocampo}
        \textbf{1036946470}
            
        \vfill
            
        \vspace{0.8cm}
            
        \Large
        Despartamento de Ingeniería Electrónica y Telecomunicaciones\\
        Universidad de Antioquia\\
        Medellín\\
        Septiembre de 2020
            
    \end{center}
\end{titlepage}

\tableofcontents

\section{Sección introductoria}
Esta es la primera sección, podemos agregar algunos elementos adicionales y todo será escrito correctamente. Más aún, si una palabra es demasiado larga y tiene que ser truncada, babel tratará de truncarla correctamente dependiendo del idioma.

\bibliographystyle{IEEEtran}
\bibliography{references}

\end{document}
